% Copyright (C) 2014-2020 by Thomas Auzinger <thomas@auzinger.name>

\documentclass[draft,final]{vutinfth} % Remove option 'final' to obtain debug information.

% Load packages to allow in- and output of non-ASCII characters.
\usepackage{lmodern}        % Use an extension of the original Computer Modern font to minimize the use of bitmapped letters.
\usepackage[T1]{fontenc}    % Determines font encoding of the output. Font packages have to be included before this line.
\usepackage[utf8]{inputenc} % Determines encoding of the input. All input files have to use UTF8 encoding.

% Extended LaTeX functionality is enables by including packages with \usepackage{...}.
\usepackage{amsmath}    % Extended typesetting of mathematical expression.
\usepackage{amssymb}    % Provides a multitude of mathematical symbols.
\usepackage{mathtools}  % Further extensions of mathematical typesetting.
\usepackage{microtype}  % Small-scale typographic enhancements.
\usepackage[inline]{enumitem} % User control over the layout of lists (itemize, enumerate, description).
\usepackage{multirow}   % Allows table elements to span several rows.
\usepackage{booktabs}   % Improves the typesettings of tables.
\usepackage{subcaption} % Allows the use of subfigures and enables their referencing.
\usepackage[ruled,linesnumbered,algochapter]{algorithm2e} % Enables the writing of pseudo code.
\usepackage[usenames,dvipsnames,table]{xcolor} % Allows the definition and use of colors. This package has to be included before tikz.
\usepackage{nag}       % Issues warnings when best practices in writing LaTeX documents are violated.
\usepackage{todonotes} % Provides tooltip-like todo notes.
\usepackage{hyperref}  % Enables cross linking in the electronic document version. This package has to be included second to last.
\usepackage[acronym,toc]{glossaries} % Enables the generation of glossaries and lists fo acronyms. This package has to be included last.

% Define convenience functions to use the author name and the thesis title in the PDF document properties.
\newcommand{\authorname}{Forename Surname} % The author name without titles.
\newcommand{\thesistitle}{Title of the Thesis} % The title of the thesis. The English version should be used, if it exists.

% Set PDF document properties
\hypersetup{
    pdfpagelayout   = TwoPageRight,           % How the document is shown in PDF viewers (optional).
    linkbordercolor = {Melon},                % The color of the borders of boxes around crosslinks (optional).
    pdfauthor       = {\authorname},          % The author's name in the document properties (optional).
    pdftitle        = {\thesistitle},         % The document's title in the document properties (optional).
    pdfsubject      = {Subject},              % The document's subject in the document properties (optional).
    pdfkeywords     = {a, list, of, keywords} % The document's keywords in the document properties (optional).
}

\setpnumwidth{2.5em}        % Avoid overfull hboxes in the table of contents (see memoir manual).
\setsecnumdepth{subsection} % Enumerate subsections.

\nonzeroparskip             % Create space between paragraphs (optional).
\setlength{\parindent}{0pt} % Remove paragraph identation (optional).

\makeindex      % Use an optional index.
\makeglossaries % Use an optional glossary.
%\glstocfalse   % Remove the glossaries from the table of contents.

% Set persons with 4 arguments:
%  {title before name}{name}{title after name}{gender}
%  where both titles are optional (i.e. can be given as empty brackets {}).
\setauthor{Pretitle}{\authorname}{Posttitle}{female}
\setadvisor{Pretitle}{Forename Surname}{Posttitle}{male}

% For bachelor and master theses:
\setfirstassistant{Pretitle}{Forename Surname}{Posttitle}{male}
\setsecondassistant{Pretitle}{Forename Surname}{Posttitle}{male}
\setthirdassistant{Pretitle}{Forename Surname}{Posttitle}{male}

% For dissertations:
\setfirstreviewer{Pretitle}{Forename Surname}{Posttitle}{male}
\setsecondreviewer{Pretitle}{Forename Surname}{Posttitle}{male}

% For dissertations at the PhD School and optionally for dissertations:
\setsecondadvisor{Pretitle}{Forename Surname}{Posttitle}{male} % Comment to remove.

% Required data.
\setregnumber{0123456}
\setdate{01}{01}{2001} % Set date with 3 arguments: {day}{month}{year}.
\settitle{\thesistitle}{Titel der Arbeit} % Sets English and German version of the title (both can be English or German). If your title contains commas, enclose it with additional curvy brackets (i.e., {{your title}}) or define it as a macro as done with \thesistitle.
\setsubtitle{Optional Subtitle of the Thesis}{Optionaler Untertitel der Arbeit} % Sets English and German version of the subtitle (both can be English or German).

% Select the thesis type: bachelor / master / doctor / phd-school.
% Bachelor:
\setthesis{bachelor}
%
% Master:
%\setthesis{master}
%\setmasterdegree{dipl.} % dipl. / rer.nat. / rer.soc.oec. / master
%
% Doctor:
%\setthesis{doctor}
%\setdoctordegree{rer.soc.oec.}% rer.nat. / techn. / rer.soc.oec.
%
% Doctor at the PhD School
%\setthesis{phd-school} % Deactivate non-English title pages (see below)

% For bachelor and master:
\setcurriculum{Media Informatics and Visual Computing}{Medieninformatik und Visual Computing} % Sets the English and German name of the curriculum.

% For dissertations at the PhD School:
\setfirstreviewerdata{Affiliation, Country}
\setsecondreviewerdata{Affiliation, Country}


\begin{document}

\frontmatter % Switches to roman numbering.
% The structure of the thesis has to conform to the guidelines at
%  https://informatics.tuwien.ac.at/study-services

\addtitlepage{naustrian} % German title page (not for dissertations at the PhD School).
\addtitlepage{english} % English title page.
\addstatementpage

\begin{danksagung*}
\todo{Ihr Text hier.}
\end{danksagung*}

\begin{acknowledgements*}
\todo{Enter your text here.}
\end{acknowledgements*}

\begin{kurzfassung}
\todo{Ihr Text hier.}
\end{kurzfassung}

\begin{abstract}
\todo{Enter your text here.}
\end{abstract}

% Select the language of the thesis, e.g., english or naustrian.
\selectlanguage{english}

% Add a table of contents (toc).
\tableofcontents % Starred version, i.e., \tableofcontents*, removes the self-entry.

% Switch to arabic numbering and start the enumeration of chapters in the table of content.
\mainmatter

\chapter{Introduction}
\todo{Enter your text here.}

\chapter{Additional Chapter}
\todo{Enter your text here.}

% Remove following line for the final thesis.
%% intro.tex
%% Copyright (C) 2014-2017 by Thomas Auzinger <thomas@auzinger.name>
%
% This work may be distributed and/or modified under the
% conditions of the LaTeX Project Public License, either version 1.3
% of this license or (at your option) any later version.
% The latest version of this license is in
%   http://www.latex-project.org/lppl.txt
% and version 1.3 or later is part of all distributions of LaTeX
% version 2005/12/01 or later.
%
% This work has the LPPL maintenance status `maintained'.
%
% The Current Maintainer of this work is Thomas Auzinger.
%
% This work consists of the files vutinfth.dtx and vutinfth.ins
% and the derived file vutinfth.cls.
% This work also consists of the file intro.tex.


\newacronym{ctan}{CTAN}{Comprehensive TeX Archive Network}
\newacronym{faq}{FAQ}{Frequently Asked Questions}
\newacronym{pdf}{PDF}{Portable Document Format}
\newacronym{svn}{SVN}{Subversion}
\newacronym{wysiwyg}{WYSIWYG}{What You See Is What You Get}

\newglossaryentry{texteditor}
{
  name={editor},
  description={A text editor is a type of program used for editing plain text files.}
}

\chapter{Introduction to \LaTeX}

Since \LaTeX\ is widely used in academia and industry, there exists a plethora of freely accessible introductions to the language.
Reading through the guide at \url{https://en.wikibooks.org/wiki/LaTeX} serves as a comprehensive overview for most of the functionality and is highly recommended before starting with a thesis in \LaTeX.

\section{Installation}

A full \LaTeX\ distribution\index{distribution} consists of not only of the binaries that convert the source files to the typeset documents, but also of a wide range of packages and their documentation.
Depending on the operating system, different implementations are available as shown in Table~\ref{tab:distrib}.
\textbf{Due to the large amount of packages that are in everyday use and due to their high interdependence, it is paramount to keep the installed distribution\index{distribution} up to date.}
Otherwise, obscure errors and tedious debugging ensue.

\begin{table}
  \centering
  \begin{tabular}{cccc}
    \toprule
    Distribution & Unix         & Windows      & MacOS        \\
    \midrule
    TeX Live     & \textbf{yes} & yes          & (yes)        \\
    MacTeX       & no           & no           & \textbf{yes} \\
    MikTeX       & no           & \textbf{yes} & no           \\
    \bottomrule
  \end{tabular}
  \caption{\TeX/\LaTeX\ distributions for different operating systems. Recomended choice in \textbf{bold}.}
  \label{tab:distrib} % \label has to be placed AFTER \caption to produce correct cross-references.
\end{table}

\section{Editors}

A multitude of \TeX\ \glspl{texteditor} are available differing in their editing models, their supported operating systems and their feature sets.
A comprehensive overview of \glspl{texteditor} can be found at the Wikipedia page  \url{https://en.wikipedia.org/wiki/Comparison_of_TeX_editors}.
TeXstudio (\url{http://texstudio.sourceforge.net/}) is recommended.
Most editors support the scrolling the typeset preview document to a location in the source document by \verb|Ctrl| clicking the location in the source document.

\section{Compilation}

Modern editors usually provide the compilation programs to generate \gls{pdf} documents and for most \LaTeX\ source files, this is sufficient.
More advanced \LaTeX\ functionality, such as glossaries and bibliographies, needs additional compilation steps, however.
It is also possible that errors in the compilation process invalidate intermediate files and force subsequent compilation runs to fail.
It is advisable to delete intermediate files (\verb|.aux|, \verb|.bbl|, etc.), if errors occur and persist.
All files that are not generated by the user are automatically regenerated.
To compile the current document, the steps as shown in Table~\ref{tab:compile} have to be taken.


\begin{table}
  \centering
  \begin{tabular}{rl}
    \toprule
    & Description \\
    \midrule
    1 & Scan for refs, toc/lof/lot/loa items and cites \\
    2 & Build the bibliography     \\
    3 & Link refs and build the toc/lof/lot/loa \\
    4 & Link the bibliography \\
    5 & Build the glossary \\
    6 & Build the acronyms \\
    7 & Build the index \\
    8 & Link the glossary, acronyms, and the index \\
    9 & Link the bookmarks \\
    \midrule
    & Command \\
    \midrule
    1 & \verb|pdflatex.exe  example| \\
    2 & \verb|bibtex.exe    example| \\
    3 & \verb|pdflatex.exe  example| \\
    4 & \verb|pdflatex.exe  example| \\
    5 & \verb|makeindex.exe -t example.glg -s example.ist| \\
      & \verb|              -o example.gls example.glo| \\
    6 & \verb|makeindex.exe -t example.alg -s example.ist| \\
      & \verb|              -o example.acr example.acn| \\
    7 & \verb|makeindex.exe -t example.ilg -o example.ind example.idx| \\
    8 & \verb|pdflatex.exe  example| \\
    9 & \verb|pdflatex.exe  example| \\
    \bottomrule
  \end{tabular}
  \caption{Compilation steps for this document. The following abbreviations were used: table of contents (toc), list of figures (lof), list of tables (lot), list of algorithms (loa).}
  \label{tab:compile} % \label has to be placed AFTER \caption to produce correct cross-references.
\end{table}


\section{Basic Functionality}

In this section, various examples are given of the fundamental building blocks used in a thesis.
Many \LaTeX\ commands have a rich set of options that can be supplied as optional arguments.
The documentation of each command should be consulted to get an impression of the full spectrum of its functionality.

\subsection{Floats}

Two main categories of page elements can be differentiated in the usual \LaTeX\ workflow: \textit{(i)} the main stream of text and \textit{(ii)} floating containers that are positioned at convenient positions throughout the document.
In most cases, tables, plots, and images are put into such containers since they are usually positioned at the top or bottom of pages.
These are realized by the two environments \verb|figure| and \verb|table|, which also provide functionality for cross-referencing (see Table~\ref{tab:intro} and Figure~\ref{fig:intro}) and the generation of corresponding entries in the list of figures and the list of tables.
Note that these environments solely act as containers and can be assigned arbitrary content.

\subsection{Tables}

A table in \LaTeX\ is created by using a \verb|tabular| environment or any of its extensions, e.g., \verb|tabularx|.
The commands \verb|\multirow| and \verb|\multicolumn| allow table elements to span multiple rows and columns.

\begin{table}[h] % placement specifier
  \centering
  \begin{tabular}{lll}
    \toprule
    \multicolumn{2}{c}{Position} \\
    \cmidrule{1-2} % partial horizontal rule
    Group & Abbrev & Name \\
    \midrule
    Goalkeeper & GK & Paul Robinson \\
    \midrule
    \multirow{4}{*}{Defenders} & LB & Lucus Radebe \\
                               & DC & Michael Duburry \\
                               & DC & Dominic Matteo \\
                               & RB & Didier Domi \\
    \midrule
    \multirow{3}{*}{Midfielders} & MC & David Batty \\
                                 & MC & Eirik Bakke \\
                                 & MC & Jody Morris \\
    \midrule
    Forward & FW & Jamie McMaster \\
    \midrule
    \multirow{2}{*}{Strikers} & ST & Alan Smith \\
                              & ST & Mark Viduka \\
    \bottomrule
  \end{tabular}
  \caption{Adapted example from the \LaTeX guide at \url{https://en.wikibooks.org/wiki/LaTeX/Tables}. This example uses rules specific to the \texttt{booktabs} package and employs the multi-row functionality of the \texttt{multirow} package.}
  \label{tab:intro} % \label has to be placed AFTER \caption to produce correct cross-references.
\end{table}

\subsection{Images}

An image is added to a document via the \verb|\includegraphics| command as shown in Figure~\ref{fig:intro}.
The \verb|\subcaption| command can be used to reference subfigures, such as Figure~\ref{fig:intro:full width} and~\ref{fig:intro:half width}.

\begin{figure}[h]
  \centering
  \begin{subfigure}[b]{0.45\columnwidth}
    \centering
    \includegraphics[width=\textwidth]{TU_INF_Logo_gray}
    \subcaption{The header logo at text width.}
    \label{fig:intro:full width}
  \end{subfigure}
  \begin{subfigure}[b]{0.45\columnwidth}
    \centering
    \includegraphics[width=0.5\textwidth]{TU_INF_Logo_gray}
    \subcaption{The header logo at half the text width.}
    \label{fig:intro:half width}
  \end{subfigure}
  \caption{The header logo at different sizes.}
  \label{fig:intro} % \label has to be placed AFTER \caption (or \subcaption) to produce correct cross-references.
\end{figure}

\subsection{Mathematical Expressions}

One of the original motivation to create the \TeX\ system was the need for mathematical typesetting.
To this day, \LaTeX\ is the preferred system to write math-heavy documents and a wide variety of functions aids the author in this task.
A mathematical expression can be inserted inline as $\sum_{n=1}^{\infty} \frac{1}{n^2} = \frac{\pi^2}{6}$ outside of the text stream as \[ \sum_{n=1}^{\infty} \frac{1}{n^2} = \frac{\pi^2}{6} \] or as numbered equation with
\begin{equation}
\sum_{n=1}^{\infty} \frac{1}{n^2} = \frac{\pi^2}{6}.
\end{equation}

\subsection{Pseudo Code}

The presentation of algorithms can be achieved with various packages; the most popular are \verb|algorithmic|, \verb|algorithm2e|, \verb|algorithmicx|, or \verb|algpseudocode|.
An overview is given at \url{https://tex.stackexchange.com/questions/229355}.
An example of the use of the \verb|alogrithm2e| package is given with Algorithm~\ref{alg:gauss-seidel}.

\begin{algorithm}
  \SetKw{BreakFor}{break for}
  \KwIn{A scalar~$\epsilon$, a matrix $\mathbf{A} = (a_{ij})$, a vector $\vec{b}$, and an initial vector $\vec{x}^{(0)}$}
  \KwOut{$\vec{x}^{(n)}$ with $\mathbf{A} \vec{x}^{(n)} \approx \vec{b}$}
  \For{$k\leftarrow 1$ \KwTo maximum iterations}
  {
     \For{$i\leftarrow 1$ \KwTo $n$}
     {
        $x_i^{(k)} = \frac{1}{a_{ii}} \left(b_i-\sum_{j<i} a_{ij} x_j^{(k)} - \sum_{j>i} a_{ij} x_j^{(k-1)} \right)$\;
     }
     \If{$\lvert\vec{x}^{(k)}-\vec{x}^{(k-1)}\rvert < \epsilon$}
     {\BreakFor\;}
  }
  \Return{$\vec{x}^{(k)}$\;}
  \caption{Gauss-Seidel}
  \label{alg:gauss-seidel} % \label has to be placed AFTER \caption to produce correct cross-references.
\end{algorithm}

\section{Bibliography}

The referencing of prior work is a fundamental requirement of academic writing and well supported by \LaTeX.
The \textsc{Bib}\TeX\ reference management software is the most commonly used system for this purpose.
Using the \verb|\cite| command, it is possible to reference entries in a \verb|.bib| file out of the text stream, e.g., as~\cite{Turing1936}.
The generation of the formatted bibliography needs a separate execution of \verb|bibtex.exe| (see Table~\ref{tab:compile}).

\section{Table of Contents}

The table of contents is automatically built by successive runs of the compilation, e.g., of \verb|pdflatex.exe|.
The command \verb|\setsecnumdepth| allows the specification of the depth of the table of contents and additional entries can be added to the table of contents using \verb|\addcontentsline|.
The starred versions of the sectioning commands, i.e., \verb|\chapter*|, \verb|\section*|, etc., remove the corresponding entry from the table of contents.

\section{Acronyms / Glossary / Index}

The list of acronyms, the glossary, and the index need to be built with a separate execution of \verb|makeindex| (see Table~\ref{tab:compile}).
Acronyms have to be specified with \verb|\newacronym| while glossary entries use \verb|\newglossaryentry|.
Both are then used in the document content with one of the variants of \verb|\gls|, such as \verb|\Gls|, \verb|\glspl|, or \verb|\Glspl|.
Index items are simply generated by placing \verb|\index|\marg{entry} next to all the words that correspond to the index entry \meta{entry}.
Note that many enhancements exist for these functionalities and the documentation of the \verb|makeindex| and the \verb|glossaries| packages should be consulted.

\section{Tips}

Since \TeX\ and its successors do not employ a \gls{wysiwyg} editing scheme, several guidelines improve the readability of the source content:
\begin{itemize}
\item Each sentence in the source text should start with a new line.
      This helps not only the user navigation through the text, but also enables revision control systems (e.g. \gls{svn}, Git) to show the exact changes authored by different users.
      Paragraphs are separated by one (or more) empty lines.
\item Environments, which are defined by a matching pair of \verb|\begin{name}| and \verb|\end{name}|, can be indented by whitespace to show their hierarchical structure.
\item In most cases, the explicit use of whitespace (e.g. by adding \verb|\hspace{4em}| or \verb|\vspace{1.5cm}|) violates typographic guidelines and rules.
      Explicit formatting should only be employed as a last resort and, most likely, better ways to achieve the desired layout can be found by a quick web search.
\item The use of bold or italic text is generally not supported by typographic considerations and the semantically meaningful \verb|\emph{|\texttt{$\dots$}\verb|}| should be used.
\end{itemize}

The predominant application of the \LaTeX\ system is the generation of \gls{pdf} files via the \textsc{Pdf}\LaTeX\ binaries.
In the current version of \textsc{Pdf}\LaTeX, it is possible that absolute file paths and user account names are embedded in the final \gls{pdf} document.
While this poses only a minor security issue for all documents, it is highly problematic for double blind reviews.
The process shown in Table~\ref{tab:ps2pdf} can be employed to strip all private information from the final \gls{pdf} document.

\begin{table}[h]
  \centering
  \begin{tabular}{rl}
  \toprule
  & Command \\
  \midrule
  1 & Rename the \gls{pdf} document \verb|final.pdf| to \verb|final.ps|. \\
  2 & Execute the following command: \\
    & \verb|ps2pdf -dPDFSETTINGS#/prepress ^| \\
    & \verb| -dCompatibilityLevel#1.4 ^| \\
    & \verb| -dAutoFilterColorImages#false ^| \\
    & \verb| -dAutoFilterGrayImages#false ^| \\
    & \verb| -dColorImageFilter#/FlateEncode ^| \\
    & \verb| -dGrayImageFilter#/FlateEncode ^| \\
    & \verb| -dMonoImageFilter#/FlateEncode ^| \\
    & \verb| -dDownsampleColorImages#false ^| \\
    & \verb| -dDownsampleGrayImages#false ^| \\
    & \verb| final.ps final.pdf| \\
  \bottomrule
  \end{tabular}

  On Unix-based systems, replace \verb|#| with \verb|=| and \verb|^| with \verb|\|.
  \caption{Anonymization of \gls{pdf} documents.}
  \label{tab:ps2pdf}
\end{table}

\section{Resources}

\subsection{Useful Links}

In the following, a listing of useful web resources is given.
\begin{description}
\item[\url{https://en.wikibooks.org/wiki/LaTeX}] An extensive wiki-based guide to \LaTeX.
\item[\url{http://www.tex.ac.uk/faq}] A (huge) set of \gls{faq} about \TeX\ and \LaTeX.
\item[\url{https://tex.stackexchange.com/}] The definitive user forum for non-trivial \LaTeX-related questions and answers.
\end{description}

\subsection[Comprehensive TeX Archive Network]{\gls{ctan}}

The \gls{ctan} is the official repository for all \TeX\ related material.
It can be accessed via \url{https://www.ctan.org/} and hosts (among other things) a huge variety of packages that provide extended functionality for \TeX\ and its successors.
Note that most packages contain \gls{pdf} documentation that can be directly accessed via \gls{ctan}.

In the following, a short, non-exhaustive list of relevant \gls{ctan}-hosted packages is given together with their relative path.
\begin{description}[itemsep=0ex]
\item[\href{https://www.ctan.org/pkg/algorithm2e}{algorithm2e}] Functionality for writing pseudo code.
\item[\href{https://www.ctan.org/pkg/amsmath}{amsmath}] Enhanced functionality for typesetting mathematical expressions.
\item[\href{https://www.ctan.org/pkg/amsfonts}{amssymb}] Provides a multitude of mathematical symbols.
\item[\href{https://www.ctan.org/pkg/booktabs}{booktabs}] Improved typesetting of tables.
\item[\href{https://www.ctan.org/pkg/enumitem}{enumitem}] Control over the layout of lists (\verb|itemize|, \verb|enumerate|, \verb|description|).
\item[\href{https://www.ctan.org/pkg/fontenc}{fontenc}] Determines font encoding of the output.
\item[\href{https://www.ctan.org/pkg/glossaries}{glossaries}] Create glossaries and list of acronyms.
\item[\href{https://www.ctan.org/pkg/graphicx}{graphicx}] Insert images into the document.
\item[\href{https://www.ctan.org/pkg/inputenc}{inputenc}] Determines encoding of the input.
\item[\href{https://www.ctan.org/pkg/l2tabu}{l2tabu}] A description of bad practices when using \LaTeX.
\item[\href{https://www.ctan.org/pkg/mathtools}{mathtools}] Further extension of mathematical typesetting.
\item[\href{https://www.ctan.org/pkg/memoir}{memoir}] The document class on upon which the \verb|vutinfth| document class is based.
\item[\href{https://www.ctan.org/pkg/multirow}{multirow}] Allows table elements to span several rows.
\item[\href{https://www.ctan.org/pkg/pgfplots}{pgfplots}] Function plot drawings.
\item[\href{https://www.ctan.org/pkg/pgf}{pgf/TikZ}] Creating graphics inside \LaTeX\ documents.
\item[\href{https://www.ctan.org/pkg/subcaption}{subcaption}] Allows the use of subfigures and enables their referencing.
\item[\href{https://www.ctan.org/tex-archive/info/symbols/comprehensive/}{symbols/comprehensive}] A listing of around 5000 symbols that can be used with \LaTeX.
\item[\href{https://www.ctan.org/pkg/voss-mathmode}{voss-mathmode}] A comprehensive overview of typesetting mathematics in \LaTeX.
\item[\href{https://www.ctan.org/pkg/xcolor}{xcolor}] Allows the definition and use of colors.
\end{description} % A short introduction to LaTeX.

\backmatter

% Use an optional list of figures.
\listoffigures % Starred version, i.e., \listoffigures*, removes the toc entry.

% Use an optional list of tables.
\cleardoublepage % Start list of tables on the next empty right hand page.
\listoftables % Starred version, i.e., \listoftables*, removes the toc entry.

% Use an optional list of alogrithms.
\listofalgorithms
\addcontentsline{toc}{chapter}{List of Algorithms}

% Add an index.
\printindex

% Add a glossary.
\printglossaries

% Add a bibliography.
\bibliographystyle{alpha}
\bibliography{intro}

\end{document}