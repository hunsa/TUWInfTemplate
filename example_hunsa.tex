% Copyright (C) 2014-2020 by Thomas Auzinger <thomas@auzinger.name>
% changed 2022 by Sascha Hunold

\documentclass[draft,final]{vutinfth} % Remove option 'final' to obtain debug information.

% Load packages to allow in- and output of non-ASCII characters.
\usepackage{lmodern}        % Use an extension of the original Computer Modern font to minimize the use of bitmapped letters.
\usepackage[T1]{fontenc}    % Determines font encoding of the output. Font packages have to be included before this line.
\usepackage[utf8]{inputenc} % Determines encoding of the input. All input files have to use UTF8 encoding.

% Extended LaTeX functionality is enables by including packages with \usepackage{...}.
\usepackage{amsmath}    % Extended typesetting of mathematical expression.
\usepackage{amssymb}    % Provides a multitude of mathematical symbols.
\usepackage{mathtools}  % Further extensions of mathematical typesetting.
\usepackage{microtype}  % Small-scale typographic enhancements.
\usepackage[inline]{enumitem} % User control over the layout of lists (itemize, enumerate, description).
\usepackage{multirow}   % Allows table elements to span several rows.
\usepackage{booktabs}   % Improves the typesettings of tables.
\usepackage{subcaption} % Allows the use of subfigures and enables their referencing.
\usepackage[usenames,dvipsnames,table]{xcolor} % Allows the definition and use of colors. This package has to be included before tikz.
\usepackage{nag}       % Issues warnings when best practices in writing LaTeX documents are violated.
\usepackage{todonotes} % Provides tooltip-like todo notes.
\setuptodonotes{inline}

% \usepackage[acronym,toc]{glossaries} % Enables the generation of glossaries and lists fo acronyms. This package has to be included last.
% hunsa
\usepackage[binary-units]{siunitx}
\usepackage{xspace}
\usepackage{my_macros}

% hunsa optional for algorithms (change this if you want to use other algorithms packages)
% \usepackage{algorithmicx}
%\usepackage[ruled,linesnumbered,algochapter]{algorithm2e} % Enables the writing of pseudo code.

\usepackage{algorithm}
\usepackage{algorithmicx}
\usepackage{algpseudocode}
\usepackage{listings}

\lstset{
  frame=single,
  basicstyle=\footnotesize\ttfamily,
  showstringspaces=false
}


\usepackage{hyperref}  % Enables cross linking in the electronic document version. This package has to be included second to last.

% Define convenience functions to use the author name and the thesis title in the PDF document properties.
\newcommand{\authorname}{Forename Surname} % The author name without titles.
\newcommand{\thesistitle}{Title of the Thesis} % The title of the thesis. The English version should be used, if it exists.

% Set PDF document properties
\hypersetup{
    pdfpagelayout   = TwoPageRight,           % How the document is shown in PDF viewers (optional).
    linkbordercolor = {Melon},                % The color of the borders of boxes around crosslinks (optional).
    pdfauthor       = {\authorname},          % The author's name in the document properties (optional).
    pdftitle        = {\thesistitle},         % The document's title in the document properties (optional).
    pdfsubject      = {Subject},              % The document's subject in the document properties (optional).
    pdfkeywords     = {a, list, of, keywords} % The document's keywords in the document properties (optional).
}

\setpnumwidth{2.5em}        % Avoid overfull hboxes in the table of contents (see memoir manual).
\setsecnumdepth{subsection} % Enumerate subsections.

\nonzeroparskip             % Create space between paragraphs (optional).
\setlength{\parindent}{0pt} % Remove paragraph identation (optional).


% \makeindex      % Use an optional index.
% \makeglossaries % Use an optional glossary.
%\glstocfalse   % Remove the glossaries from the table of contents.

% Set persons with 4 arguments:
%  {title before name}{name}{title after name}{gender}
%  where both titles are optional (i.e. can be given as empty brackets {}).
\setauthor{Pretitle}{\authorname}{Posttitle}{female}
\setadvisor{Pretitle}{Forename Surname}{Posttitle}{male}

% For bachelor and master theses:
\setfirstassistant{Pretitle}{Forename Surname}{Posttitle}{male}
\setsecondassistant{Pretitle}{Forename Surname}{Posttitle}{male}
\setthirdassistant{Pretitle}{Forename Surname}{Posttitle}{male}

% For dissertations:
\setfirstreviewer{Pretitle}{Forename Surname}{Posttitle}{male}
\setsecondreviewer{Pretitle}{Forename Surname}{Posttitle}{male}

% For dissertations at the PhD School and optionally for dissertations:
\setsecondadvisor{Pretitle}{Forename Surname}{Posttitle}{male} % Comment to remove.

% Required data.
\setregnumber{0123456}
\setdate{01}{01}{2001} % Set date with 3 arguments: {day}{month}{year}.
\settitle{\thesistitle}{Titel der Arbeit} % Sets English and German version of the title (both can be English or German). If your title contains commas, enclose it with additional curvy brackets (i.e., {{your title}}) or define it as a macro as done with \thesistitle.
\setsubtitle{Optional Subtitle of the Thesis}{Optionaler Untertitel der Arbeit} % Sets English and German version of the subtitle (both can be English or German).

% Select the thesis type: bachelor / master / doctor / phd-school.
% Bachelor:
\setthesis{bachelor}
%
% Master:
%\setthesis{master}
%\setmasterdegree{dipl.} % dipl. / rer.nat. / rer.soc.oec. / master
%
% Doctor:
%\setthesis{doctor}
%\setdoctordegree{rer.soc.oec.}% rer.nat. / techn. / rer.soc.oec.
%
% Doctor at the PhD School
%\setthesis{phd-school} % Deactivate non-English title pages (see below)

% For bachelor and master:
%\setcurriculum{Media Informatics and Visual Computing}{Medieninformatik und Visual Computing} % Sets the English and German name of the curriculum.
\setcurriculum{Software Engineering and Internet Computing}{Software Engineering und Internet Computing} % Sets the English and German name of the curriculum.


% For dissertations at the PhD School:
\setfirstreviewerdata{Affiliation, Country}
\setsecondreviewerdata{Affiliation, Country}


\begin{document}

\frontmatter % Switches to roman numbering.
% The structure of the thesis has to conform to the guidelines at
%  https://informatics.tuwien.ac.at/study-services

\addtitlepage{naustrian} % German title page (not for dissertations at the PhD School).
\addtitlepage{english} % English title page.
\addstatementpage

\begin{danksagung*}
  \todo{Eine Danksagung ist ausreichend. Entweder in deutscher oder in
    englischer Sprache. Die andere wird gelöscht.}
\end{danksagung*}

\begin{acknowledgements*}
\todo{Enter your text here.}
\end{acknowledgements*}

\begin{kurzfassung*}
  \todo{Die Kurzfassung (abstract) ist sehr wichtig. Umreißen Sie den
    Kontext der Arbeit. Was genau ist ie Fragestellung, die bearbeitet
    wurde. Wie wurde das Problem angegangen und was wurde
    herausgefunden? Die Kurzfassung ist ein kurzer Abriss der
    gesamten Arbeit.}
\end{kurzfassung*}

\begin{abstract*}
\todo{Enter your text here.}
\end{abstract*}

% Select the language of the thesis, e.g., english or naustrian.
\selectlanguage{english}

% Add a table of contents (toc).
\tableofcontents* % Starred version, i.e., \tableofcontents*, removes the self-entry.
\cleardoublepage % Start list of tables on the next empty right hand page.

% Use an optional list of figures.
\listoffigures % Starred version, i.e., \listoffigures*, removes the toc entry.
\cleardoublepage % Start list of tables on the next empty right hand page.

% Use an optional list of tables.
\listoftables % Starred version, i.e., \listoftables*, removes the toc entry.
\cleardoublepage % Start list of tables on the next empty right hand page.

% Use an optional list of algorithms.
\listofalgorithms
\addcontentsline{toc}{chapter}{List of Algorithms}

% Switch to arabic numbering and start the enumeration of chapters in the table of content.
\mainmatter

\chapter{Introduction}
\todo{Enter your text here.}

% Remove following line for the final thesis.
%% intro.tex


% hunsa's version of intro.tex
% Copyright (C) 2022

\chapter{Introduction to Scientific Writing}

In the following, we will give you a few hints on how to write your
thesis.

\section{Formatting Guidelines}

\subsection{Structure}
Structure your thesis properly. If you happen to have only one section
or subsection in a chapter or section, respectively, then remove the
heading.

For example, if you have
\begin{lstlisting}
Chapter 1 Introduction
1.1 Contribution
\end{lstlisting}
then change it into
\begin{lstlisting}
Chapter 1 Introduction
\end{lstlisting}

Similarly, if you have
\begin{lstlisting}
1.1 Quicksort
1.1.1 Randomized Quicksort
\end{lstlisting}
then, either you need a section with \verb|1.1.2| or remove the
heading \verb|1.1.1|.

\subsection{Capitalizing Headings}

All words (except ``a'', ``the'', ``in'', \etcet) are typically
capitalized (also conforms to guidelines of ACM and IEEE).

Example:
\begin{verbatim}
2 My Algorithm
2.1 Theoretical Foundation
2.2 NP-hardness Proof
2.3 Pseudocode of ABC Algorithm
2.4 Implementation using Java
2.5 Experimental Evaluation
\end{verbatim}

\section{Punctuation}

\subsection{Avoid Contractions}

Please do not use contractions in scientific writing, such as
``it's'', ``can't'', ``wanna'' or  ``won't''.

\subsection{Quotes}

Quoting seems to be very hard for some. When you want to refer to
someone else, you may use ``quotes''. But the quotes need to done
correctly, \ie, you start the quoted text with \verb|``| (two
backticks) and mark the end of the quote with \verb|''| (two
apostrophes). This ensures that the quotes are ``always set
correctly.''

\subsection{Footnotes}

You may want to use footnotes at times. However, if you do so, and
they happen to be at the end of the sentence, then put the period
before the footnote.\footnote{This is after the period.}

\subsection{Units}

The package \verb|siunitx| helps you with writing numbers and units.

For example, you may want to measure for \SI{100}{\micro\second} and
create file with \SI{10}{\gibi\byte} or \SI{10}{\giga\byte}, which
ever base you prefer.

\section{Floating Environments}

We usually try to align floats on top of the page, \eg,
\verb|{figure}[t]| or \verb|{table}[t]|. Sometimes it may be necessary
to modify this guidelines, \eg, on the first page, and then we can use
\verb|[h]|.

\subsection{Figures}

Images can be added with the \verb|\includegraphics| command as shown
in Figure~\ref{fig:intro}.  With \verb|\subcaption|, you can reference
subfigures, such as Figures~\ref{fig:intro:full width}
and~\ref{fig:intro:half width}.
%
\begin{figure}[h]
  \centering
  \begin{subfigure}[b]{0.45\columnwidth}
    \centering
    \includegraphics[width=\textwidth]{Logo-schwarz.pdf}
    \subcaption{The header logo at text width.}
    \label{fig:intro:full width}
  \end{subfigure}
  \begin{subfigure}[b]{0.45\columnwidth}
    \centering
    \includegraphics[width=0.5\textwidth]{Logo-schwarz.pdf}
    \subcaption{The header logo at half the text width.}
    \label{fig:intro:half width}
  \end{subfigure}
  \caption[Short caption for TOC]{The header logo at different sizes.} 
  \label{fig:intro} 
\end{figure}
%

Notice that the caption is below the figure. Please try avoiding pixel
graphics (jpeg, png) whenever possible.

\subsection{Tables}

\begin{table}[t]
  \centering
  \caption{Captions of tables are printed above.}
  \label{tab:example1}
  \begin{tabular}{ll}
    \toprule
    Thesis  & Template    \\
    \midrule
    Bachelor & \verb|example_hunsa.tex| \\
    Master  & \verb|example_hunsa.tex| \\
    PhD & \verb|example.tex| \\
    \bottomrule
  \end{tabular}
\end{table}

\tab~\ref{tab:example1} shows one example of inserting a table. Notice
that we print its caption above the table.


\subsection{Mathematical Expressions}

You can write mathematical expressions inline, such as
$\sum_{n=1}^{\infty} \frac{1}{n^2} = \frac{\pi^2}{6}$.
You may also write them in a block.
However, if you write an equation like the following one
\begin{equation*}
x = \sum_{i=1}^{n} i \quad ,
\end{equation*}
make sure that it is part of a sentence.

You can also write an equation, which you would like to reference later.
For example, the value of $y$ is defined as follows
\begin{equation}
\label{eq:ydef}
y = \sum_{i=1}^{n} 2^i \quad .
\end{equation}

Later, we may want to reference this equation. For example, our
definition of $y$ was given in Equation~\eqref{eq:ydef}.  Notice that
we have used \verb|\eqref| to reference this equation.

\subsection{Pseudo Codes and Listings}

Pseudo codes, algorithms, and other listings should also have a
caption above.

Let us give two example of how to work with algorithms and listings.

For writing algorithms, you can use the \verb|algorithmic| or
\verb|algorithmicx| package. Of course, there are also other packages
for achieving similar results. You may use other ones as well.

\begin{algorithm}[t]
\caption{Euclid’s algorithm}\label{alg:euclid}
\begin{algorithmic}[1]
  \Procedure{Euclid}{$a,b$}
  \State $r\gets a \mod b$
  \While{$r\not=0$}
  \State $a\gets b$
  \State $b\gets r$
  \State $r\gets a\bmod b$
  \EndWhile\label{euclidendwhile}
  \State \textbf{return} $b$
  \EndProcedure
\end{algorithmic}
\end{algorithm}

\alg~\ref{alg:euclid} shows an example algorithm, which was taken
directly from the documentation of the \verb|algorithmicx| package.

If you want to reference a particular line of the algorithm, you can
do that, but use \emph{Line}, \eg, in Line~\ref{euclidendwhile}, the
while loop is closed.

\begin{lstlisting}[language=C++,
  caption={Example of OpenMP program.},
  float=h,
  numbers=left]
#include <omp.h>

main ()  {

int nthreads, tid;

/* Fork a team of threads with each thread having a private tid variable */
#pragma omp parallel private(tid)
  {

  /* Obtain and print thread id */
  tid = omp_get_thread_num();
  printf("Hello World from thread = %d\n", tid);

  /* Only master thread does this */
  if (tid == 0) 
    {
    nthreads = omp_get_num_threads();
    printf("Number of threads = %d\n", nthreads);
    }

  }  /* All threads join master thread and terminate */

}
\end{lstlisting}

\section{Bibliography}

The bibliography is an important source of information for the reader.
Please ensure that each entry in the bibliography is completely
filled, \eg, each entry has a year and each conference paper has a
title of the proceedings it was published in.

Significant work on the decision problem has been done
before~\cite{Turing1936}.  Make sure that proper names and
abbreviations are surrounded with curly braces in your bib file, e.g.,
\verb|{MPI}| or \verb|{CUDA}|.

Turing~\cite{Turing1936} has done considerable work on the Entscheidungsproblem.

Terpstra~\etal~\cite{papi} showed how performance data can be collected with PAPI.

It could also happen that you will not need \etal, especially if there
are only two authors. For example, in 1972, Coffmann and
Graham~\cite{CoffmanG72} presented an optimal scheduling algorithm for
a system with two processors.

%%% Local Variables:
%%% mode: latex
%%% TeX-master: "example_hunsa"
%%% End:
 % A short introduction to LaTeX.

\chapter{Additional Chapter}
\todo{Enter your text here.}

\backmatter

% Add an index.
% hunsa: braucht man nicht
% \printindex

% Add a glossary.
% hunsa: braucht man nicht
% \printglossaries

% Add a bibliography.

% hunsa: alpha? 
% \bibliographystyle{alpha}

% hunsa: we go with IEEE/ACM here
\bibliographystyle{ieeetr}
\bibliography{intro}

\end{document}